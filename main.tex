\documentclass[a4paper,12pt]{article}
\usepackage[utf8]{inputenc}
\usepackage[LGR,T1]{fontenc}
\usepackage{alphabeta}
\usepackage{amsmath}
\usepackage{float}
\usepackage{multirow}
\usepackage{amsfonts}
\usepackage{amssymb}
\usepackage{graphicx}
\usepackage{listings}
\usepackage{xcolor}

\lstset{
  basicstyle=\ttfamily\footnotesize,
  keywordstyle=\color{blue},
  commentstyle=\color{gray},
  stringstyle=\color{red},
  showstringspaces=false,
  breaklines=true,
  frame=single,
  language=R,
  extendedchars=true,
  literate={β}{{\beta}}1
}



\title{Αναγνώριση Προτύπων \\ 1η Εργαστηριακή Άσκηση \\ Χειμερινό Εξάμηνο 2024-2025 \\ Ε.ΔE.ΜΜ}
\author{Σπανάκης Παναγιώτης-Αλέξιος (ΑΜ: 03400274)}
\date{08/11/2024}

\begin{document}

\maketitle
\section*{Βήμα 1}


\begin{table}[h!]
    \centering
    \begin{tabular}{|c|c|c|c|c|c|}
    \hline
    \textbf{Vowel} & \textbf{Speaker} & \textbf{Pitch (Hz)} & \textbf{F1 (Hz)} & \textbf{F2 (Hz)} & \textbf{F3 (Hz)} \\ \hline
    \multirow{2}{*}{α} & man & 134.06 & 777.91 & 1217.69 & 2405.49 \\ \cline{2-6} 
                       & woman & 176.84 & 859.08 & 1837.31 & 3146.39 \\ \hline
    \multirow{2}{*}{ου} & man & 130.65 & 372.17 & 1788.73 & 2327.56 \\ \cline{2-6} 
                        & woman & 184.51 & 321.15 & 1566.12 & 2631.54 \\ \hline
    \multirow{2}{*}{ι} & man & 132.04 & 387.10 & 2047.69 & 2556.83 \\ \cline{2-6} 
                       & woman & 178.49 & 368.76 & 2259.41 & 2951.24 \\ \hline
    \end{tabular}
    \caption{Measurements for vowels by male and female speakers.}
\end{table}

\textbf{Αρχικά, Σύγκριση Θεμελιώδους Συχνότητας (Pitch):}
Όπως αναμενόταν, η μέση συχνότητα της γυναικείας ομιλήτριας (Ομιλητής 2) είναι αισθητά υψηλότερη από εκείνη του ανδρικού ομιλητή (Ομιλητής 1) για όλα τα φωνήεντα. Συγκεκριμένα, για το φωνήεν \textbf{"α"} ο άντρας έχει 134 Hz ενώ η γυναίκα 177 Hz, για το \textbf{"ου"} ο άντρας έχει 131 Hz ενώ η γυναίκα 185 Hz, και για το \textbf{"ι"} ο άντρας 132 Hz έναντι της γυναίκας στα 178 Hz. Αυτή η διαφορά είναι σύμφωνη με τις γενικές φωνητικές τάσεις όπου οι γυναίκες έχουν υψηλότερες συχνότητες λόγω ανατομικών διαφορών, όπως οι μικρότερες και πιο τεντωμένες φωνητικές χορδές.

\textbf{Παρατηρήσεις για τα Formants:}
Οι τιμές των μορφάντων (F1, F2, F3) είναι επίσης υψηλότερες για τη γυναίκα ομιλήτρια συγκριτικά με τον άντρα, κάτι που αναμένεται λόγω του μικρότερου φωνητικού σωλήνα, ο οποίος επηρεάζει τις συχνότητες αντήχησης και οδηγεί σε μεγαλύτερες τιμές για τα formants. Για παράδειγμα, για το φωνήεν \textbf{"α"}, η πρώτη μορφάντις (F1) του άνδρα είναι 778 Hz, ενώ της γυναίκας είναι 859 Hz. Παρόμοια, η δεύτερη μορφάντις (F2) του άνδρα είναι 1218 Hz ενώ της γυναίκας 1837 Hz, και η τρίτη (F3) του άνδρα είναι 2405 Hz έναντι 3146 Hz για τη γυναίκα. Το ίδιο μοτίβο παρατηρείται και στα υπόλοιπα φωνήεντα.

\textbf{Ειδικές Παρατηρήσεις ανά Φωνήεν:}
\begin{itemize}
    \item Για το \textbf{"α"}, παρατηρούμε υψηλότερη F1 και F2 και στους δύο ομιλητές σε σχέση με τα φωνήεντα "ου" και "ι", κάτι που δείχνει ότι το "α" είναι πιο ανοιχτό και κεντρικό φωνήεν. Οι σημαντικά υψηλότερες τιμές F2 και F3 για τη γυναίκα ομιλήτρια πιθανώς υποδεικνύουν μια πιο πρόσθια εκφορά.
    \item Το \textbf{"ου"} εμφανίζει τη χαμηλότερη F1 και στους δύο ομιλητές, κάτι που δείχνει πως πρόκειται για ένα πιο κλειστό και οπίσθιο φωνήεν. Η διαφορά της F2 ανάμεσα στους ομιλητές ενισχύει την οπίσθια θέση του φωνήεντος και για τα δύο φύλα.
    \item Για το \textbf{"ι"}, παρατηρείται υψηλή F2, ειδικά για τη γυναίκα, κάτι που συνάδει με τον πρόσθιο, υψηλό χαρακτήρα του φωνήεντος αυτού. Η κοντινή απόσταση μεταξύ F2 και F3 στη γυναικεία φωνή μπορεί να υποδηλώνει μια πιο έντονη εκφορά.
\end{itemize}

\textbf{Γενικές Διαφορές μεταξύ των Φύλων:}
Οι διαφορές στις συχνότητες της θεμελιώδους συχνότητας και των μορφάντων που παρατηρήθηκαν είναι σύμφωνες με τις κοινές φωνητικές διακρίσεις μεταξύ ανδρικών και γυναικείων φωνών, με τον ανδρικό ομιλητή να εμφανίζει χαμηλότερες συχνότητες σε όλες τις παραμέτρους.

\section*{Βήμα 2}



\end{document}